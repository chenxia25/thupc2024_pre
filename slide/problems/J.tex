\begin{frame}{J 套娃 {by \itshape Elegia}}
	给一个序列 $a$,对每个 $k$,询问
	\begin{itemize}
	\item 把所有长为 $k$ 的子区间求 mex,对得到的答案集合再取 mex
	\end{itemize}
	的结果.


\end{frame}

\begin{frame}{解法}
	首先考虑所有里层的 mex 的数会如何出现. \pause

	如果一段区间的 mex 值是 $v$,那么这个区间首先不存在一个值域为 $v$ 的数,然后 $0,...,v-1$ 这些数都出现过. \pause

	我们首先考虑数列中所有出现过的数字 $v$,它们出现过的位置将这个序列分割成 $c_v$ 段(我们有 $\sum c_v=O(n)$),mex 值是 $v$ 的段肯定只能是其中某一段的子区间. 而且如果某一段里存在的话,那么这一整段肯定存在,所以有一个长度区间 $[d, u]$,表示 $d,...,u$ 这些长度都有作为 $v$ 的 mex 存在过.

\end{frame}


\begin{frame}{解法}
	如果处理出了这些 $[d, u]$,我们只需要扫描线一下就就可以求出要输出的所有答案了,复杂度 $O(n \log n)$. \pause

	接下来考虑如何确定这些 $[d, u]$。从小到大考虑 $v$,对于每个 $i$ 我们维护一个最小的数 $p_i$ 表示 $[i, p_i]$ 里出现过所有 $0,...,v-1$ 的数, 那么 $p_i$ 这个阶梯型可以用线段树维护. \pause

	查询被 $v$ 分割出的一段 $[l, r]$ 里最小的区间时, 我们先想办法二分出最大的 $i$ 满足 $p_i < r$,然后用线段树顺带维护出 $p_i - i + 1$ 的最小值,对此区间查询. \pause

	复杂度 $O(n \log n)$.


\end{frame}
