\frame
{
  \frametitle{D 多折较差验证 {by \itshape SpiritualKhorosho}}

  	对于 $w=w_1 w_2 \cdots w_L \in \left\{\land, \lor\right\}^L (L\ge 0), \alpha \in \left\{\land, \lor\right\}$,称具有形式 $w + \alpha + \bar{w}$ 的折痕序列是可以沿着 $\alpha$ 恰好重合地折叠的,其中
	$\bar{w} = \bar{w}_L \bar{w}_{L-1} \cdots \bar{w}_1$,而对单个字符有 $\bar{\land} = \lor, \bar{\lor} = \land$。

	对于任意的折痕序列 $w = w_1 w_2 \cdots w_M$,称 $w$ 可以在 $w_k$ 处折叠,当且仅当:
	\begin{itemize}
		\item 若 $k\le (M+1)/2$,则 $w_1 \cdots w_{2k+1}$ 可以沿着 $w_k$ 恰好重合 地折叠;折叠后得到 $w_{k+1} \cdots w_L$,不对称程度 $L - 2k - 1$。
		\item 若 $k> (M+1)/2$,则 $w_{2k-L} \cdots w_L$ 可以沿着 $w_k$ 恰好重合地折叠;折叠后得到 $w_1 \cdots w_{k-1}$,不对称程度 $2k-L-1$。
	\end{itemize}

	给出折痕序列 $s_1 s_2 \cdots s_N \in \left\{\land, \lor\right\}$,求出将该序列折叠成空串的最小操作次数,并求出在此前提下最小的不对称程度之和。

	$1\le N\le 5000.$
}

\begin{frame}{分析}
	
	题目中描述的折叠过程具有最优子结构的性质,这启发我们使用区间 DP 进行求解.

	记 $Check(l, r, k) = 1$ 表示 $s_l \cdots s_r$ 可以在 $s_k$ 处折叠,$Check(l, r, k) = 0$ 表示不可以这样折叠;$f(l,r) =(x,y)$(此处有序数对根据字典序比较)表示对 $s_l, \cdots, s_r$ 进行折叠时,最小的折叠次数为 $x$,且此时最小的不对称程度为 $y$.

	对于每个 $f(l,r)$,直接枚举$k$ 并判断是否 $Check(l,r,k)=1$ 再进行转移是 $O\left(N^3\right)$ 的,显然不能通过本题.

\end{frame}

\begin{frame}{优化转移}
	
	为此,考虑优化枚举 $k$ 的过程:从枚举所有 $k$ 改为只枚举满足重合要求的 $k$。

	为了实现这个优化,需要分别对每个 $i$ 做左端点或右端点的情况,求出有哪些 $k$ 恰好可以覆盖到端点 $i$。计算 $f(l,r)$ 时,对区间左侧枚举可以覆盖到 $l$ 的折痕 $k$,使用 $f(k+1,r)$ 进行转移;对区间右侧枚举可以覆盖到 $r$ 的折痕 $k$,使用 $f(l,k-1)$ 进行转移。\pause

	然后你会发现,你可以通过本题。

\end{frame}


\begin{frame}{复杂度?}
	
	优化后的做法的复杂度分析比较复杂\sout{,反正出题验题人都不会}。与回文串及回文半径不同的是,这里取反后对称的规定使得有效的转移次数不会很多。例如,如果在一棵完全二叉树中任意结点的左右子树都是反对称的,那它的中序遍历可以将转移次数卡到 $O\left(N^2 log N\right)$。部分测试数据是根据这个思路生成的,但出题人及验题人不确定是否有更强且规则较为简单的数据形态.

	如果有选手可以证明本题的转移次数是 $O(N^2 \log N)$ 的,或者举出不满足这一复杂度的数据构造方式,都欢迎提出.

\end{frame}