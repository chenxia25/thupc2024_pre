\frame
{
  \frametitle{A 排序大师 {by \itshape Itst}}
	用最少的 "段交换" 操作将长度为 $n$ 的排列排序。
	
	"段交换" 操作为以下操作:将排列分成五段,其中第二段和第四段不能为空、其他段可以为空,交换第二段和第四段再拼回去。

	$n\le 2000$.
}

\begin{frame}{正解}

	需要最小化交换次数,意味着我们同时需要上界和下界的分析.

	下界分析的常用办法是图模型,最常见的是 $p_i\rightarrow i$ 的模型. 但对于这个操作来说它不适用,因为一次交换会导致图上很多的边产生改变. \pause

	这提示我们需要考虑这样的图模型:在这个模型下,一次段交换操作只会改变其中常数条边.

	再注意到,一次交换中形如 $p_i\rightarrow p_{i+1}$ 的边只会改变最多四条,符合我们的要求,但这个模型并不能直接用于下界分析,因为目标状态是一条特殊的链,不能用环数分析.

\end{frame}

\begin{frame}{正解}

	我们做一些简单的调整:在排列最前面加入一个 $0$、最后加入一个 $n+1$,然后加入 $p_i\rightarrow p_{i+1}$ 的边。此时终态是一个 ${n+1}$ 个自环构成的图,符合“单次操作改变边数少”,同时环数相关的下界分析可以利用了.

	此时我们来考虑一次操作究竟会干什么.

\end{frame}

\begin{frame}{正解}
	不妨假设第三段非空,此时交换前后的状况如下:

	(插入图片ing)

	最初的四条边


$p_{i-1}+1\rightarrow p_i, p_j+1\rightarrow p_{j+1}, p_{k-1}+1\rightarrow p_k, p_l+1\rightarrow p_{l+1}$

 	被改为

$p_{i-1}+1\rightarrow p_k, p_l+1\rightarrow p_{j+1}, p_{k-1}+1\rightarrow p_i, p_j+1\rightarrow p_{l+1}$

	看起来还是画图比较直观(

\end{frame}

\begin{frame}{正解}
To Be Continue...
\end{frame}